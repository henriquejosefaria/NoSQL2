\section{Questão 3}


As diferenças entre uma base de dados relacional e uma base de dados documental são as a seguir enunciadas:
\begin{itemize}
 \item Tabelas\hfill\newline
 \par Numa  base de dados relacional os dados referentes a uma entidade podem estar distribuidos por várias tabelas sendo que cada tabela contêm colunas que correspondem a atributos, linhas que correspondem cada uma a um registo e chaves estrangeiras e primárias para estabelecerem relações entre si. Já numa base de dados documental não se usam tabelas pra armazenar informação sobre uma entidade, em vez disso os dados relacionados como uma entidade são armazenados num único documento e todos os dados associados á mesma são aninhados dentro do mesmo.\hfill\newline
 \item Esquemas\hfill\newline
 \par Numa base de dados relacional antes de se inserir qualquer dado é criado um esquema lógico. Já nas bases de dados documentais podemos carregar os dados sem utilizarmos um esquema pré-definido.\hfill\newline
 \item Escalabilidade\hfill\newline
 \par Enquanto as bases de dados relacionais são adequadas a escalar verticalmente as bases de dados documentais podem ser dimensionadas horizontalmente (sharding) o que permite o armazenamento da base de dados em milhares de computadores sem comprometer o desempenho.\hfill\newline
 \item Relacionamentos\hfill\newline
 \par Como já foi referido anteriormente, os dados armazenados em tabelas nas bases de dados relacionais relacionam-se utilizando chaves estrangeiras. No entanto, nas bases de dados documentais não existem chaves estrangeiras, em vez disso os dados relacionados com uma entidade são aninhados dentro do documento da mesma.\hfill\newline
\item Linguagem \hfill\newline
 \par As bases de dados relacionais utilizam SQL como linguagem de interrogação da mesma. No MongoDB é utilizada uma sintaxe própria.
\end{itemize}
